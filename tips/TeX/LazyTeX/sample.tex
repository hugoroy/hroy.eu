\documentclass[english,]{amsart}
\usepackage{lmodern}
\usepackage{amssymb,amsmath}
\usepackage{ifxetex,ifluatex}
\usepackage{fixltx2e} % provides \textsubscript
\ifnum 0\ifxetex 1\fi\ifluatex 1\fi=0 % if pdftex
  \usepackage[T1]{fontenc}
  \usepackage[utf8]{inputenc}
\else % if luatex or xelatex
  \usepackage{fontspec}
  \defaultfontfeatures{Ligatures=TeX,Scale=MatchLowercase}
  \newcommand{\euro}{€}
    \setmainfont[]{Vendome Amateur}
\fi
% use upquote if available, for straight quotes in verbatim environments
\IfFileExists{upquote.sty}{\usepackage{upquote}}{}
% use microtype if available
\IfFileExists{microtype.sty}{%
\usepackage{microtype}
\UseMicrotypeSet[protrusion]{basicmath} % disable protrusion for tt fonts
}{}
\usepackage[]{hyperref}

\PassOptionsToPackage{usenames,dvipsnames}{color} % color is loaded by hyperref

\hypersetup{unicode=true,            pdftitle={How to write in Lazy~instead of strict},            pdfauthor={Hugo Roy},            pdfborder={0 0 0},            breaklinks=true}
            
\urlstyle{same}  % don't use monospace font for urls
\ifnum 0\ifxetex 1\fi\ifluatex 1\fi=0 % if pdftex
  \usepackage[shorthands=off,,main=english]{babel}
  
\else
  \usepackage{polyglossia}
  \setmainlanguage[]{english}
\fi
\usepackage{graphicx,grffile}
\makeatletter
\def\maxwidth{\ifdim\Gin@nat@width>\linewidth\linewidth\else\Gin@nat@width\fi}
\def\maxheight{\ifdim\Gin@nat@height>\textheight\textheight\else\Gin@nat@height\fi}
\makeatother
% Scale images if necessary, so that they will not overflow the page
% margins by default, and it is still possible to overwrite the defaults
% using explicit options in \includegraphics[width, height, ...]{}
\setkeys{Gin}{width=\maxwidth,height=\maxheight,keepaspectratio}
\setlength{\parindent}{0pt}
\setlength{\parskip}{6pt plus 2pt minus 1pt}
\setlength{\emergencystretch}{3em}  % prevent overfull lines
\providecommand{\tightlist}{%
  \setlength{\itemsep}{0pt}\setlength{\parskip}{0pt}}
\setcounter{secnumdepth}{0}

\title{How to write in Lazy\TeX~instead of strict \LaTeX}
\author{Hugo Roy}
\date{\today}
\addto\captionsenglish{%
\renewcommand{\abstractname}{\textbf{TL;DR}} %
} \usepackage{varioref}
\newenvironment{toggleable}{\begin{minipage}\small}{\end{minipage}}


\begin{document}
\maketitle
\begin{abstract}
Use Pandoc, write your document in a combination of YAML, Markdown and,
when you need it, inline LaTeX. Read Pandoc's
\href{http://pandoc.org/README.html}{README}.
\bigskip

\noindent \textbf{Keywords.}
LaTeX~-- Pandoc~-- YAML~-- Markdown
\end{abstract}




TeX is awesome. LaTeX was made to make it easier to use TeX and produce
high-quality documents.

Still, there are two downsides with using LaTeX:

\begin{enumerate}
\def\labelenumi{\arabic{enumi}.}
\tightlist
\item
  the source of your document is a bit
  \hyperdef{}{point1}{\label{point1}}{cryptic for people who aren't used
  to source code}
\item
  TeX was designed for paper as the output and thus comes with its
  limitations.
\end{enumerate}

Today, most LaTeX documents end up as PDF and/or printed on paper (which
is \emph{kind of} the same). This is nice, but PDF and paper are not
mediums that enable others to co-edit the text (unless they can work
with the LaTeX source, but most people won't learn that,
see\label{above} \hyperref[point1]{point 1 above}).

This is especially sad because LaTeX is not only able to produce awesome
typesets, it's also able to produce part of the content of the document,
thanks to a myriad of packages that you can use in LaTeX (for instance,
varioref).

\begin{toggleable}
For example, the [varioref] package is the best program I've seen to
make automated references to another point in a document. Using
varioref, LaTeX is able to print something like: “see section 3, on
the facing page” automatically (or “… on the next page” if the
document isn't supposed to be printed twosides aka *recto verso*).

[varioref]: http://ctan.org/pkg/varioref

That's great, but that only makes sense for documents in pages, like
paper or PDF. It does not make sense for a document in HTML that will
be a single “web page” (of course, we could also emulate pages in HTML
but, seriously, why are people doing that?) although it still makes
sense to be able to refer to another part of the doc (like I did
\vpageref[above]{above}.)
\end{toggleable}

So what I'm doing these days is mostly \textbf{Lazy\TeX}.

\section{What's LazyTeX?}\label{whats-lazytex}

LazyTeX is a way to use TeX that is lazy, but has the potential to
overcome the two donwsides of using strict LaTeX that I just described.

Mainly, LazyTeX is just a funny name I have given to the combination of
Markdown, YALM and inline LaTeX, that can be used through Pandoc in
order to produce beautiful LaTeX PDF.

The upside to doing this, is that the source is way more legible for
people like LibreOffice or Microsoft Office users, and the output will
not necessarily be PDF but, in some cases, could as well be HTML or
plain text, or something else.

Why's this lazy? There are two reasons to this:

First, the markdown syntax is lazier than the latex syntax. For
instance, a list in markdown is as simple as writing:

\begin{verbatim}
    This is some text.

    - This is the first item of a list
    - This is the second item of a list

    This is some text.
\end{verbatim}

whereas a list in LaTeX cannot really be more simple than:

\begin{verbatim}
    This is some text.

    \begin{itemize}
    \item This is the first item of a list
    \item This is the second item of a list
    \end{itemize}

    This is some text.
\end{verbatim}

You get the idea. Sometimes, even documents that I only need as PDF, for
which I could use plain, strict LaTeX --- I today prefer to write them
in a combination of YAML, Markdown and inline LaTeX --- that means, what
I call from now on LazyTeX.

However, lazy also has a downside. Mainly, if I mike a mistake in the
source file, there are more risks of producing a PDF with the mistake
showing in plain sight, rather than having a compilation error.

When I do a mistake in a LaTeX file, usually the compilation will give
me an error and not produce the result. Thus, the error flags me that I
need to fix something.

However, when I do a mistake in a LazyTeX file (for instance, misplacing
a list inside a list because of wrong indention, or misplacing an
asterisk that's supposed to make something bold) --- in such cases, the
LazyTeX file might compile correctly and will just print the mistake. So
I may need to review the PDF more thoroughly, which can be cumbersome
for long documents. So, in some cases, maybe LazyTeX should be avoided
and strict LaTeX prefered.

\section{How does it work?}\label{how-does-it-work}

Pandoc is what makes this possible. Pandoc has its own Markdown variant,
which enables Markdown to be a bigger subset of HTML than the
``vanilla'' Markdown is. But Pandoc also has some neat tricks that makes
Markdown an interesting source for LaTeX. For instance, the
\texttt{pandoc-citeproc} program that's shipped with Pandoc enables you
to use the bibliography engines of LaTeX.

Pandoc also parses YAML data, which you can then use to generate parts
of your LaTeX document, especially the preamble.

Pandoc also allows you to have inline LaTeX, meaning you can write some
LaTeX inside your markdown and Pandoc will work it out. (Although this
has some limitations).

Obviously, one of the biggest upside of Pandoc, is the ability to
convert documents from one format to another.

\section{What's not working?}\label{whats-not-working}

The problem is that Pandoc's LaTeX ``reader'' isn't a full LaTeX parser
(yet). So the markdown+inlineLaTeX combination may cause issues for
non-LaTeX outputs.

So be careful with some commands.

See the sample document and sample PDF to see some of the limitations.

My solution right now, is to add another layer of complexity, to make
things worse: I use custom directives in Emacs'
\href{http://joostkremers.github.io/pandoc-mode/}{Pandoc-mode}.







\end{document}
